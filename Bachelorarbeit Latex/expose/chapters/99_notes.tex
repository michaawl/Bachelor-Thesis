\pagebreak
%%%%%%%%%%%%%%%%%%%%%%%%%%%%%%%%%%%%%%%%%%%%%%%%%%%%%%%%%%%%%%%%%%%%%%%%%%%%%
\TODO{\section{INTERNAL NOTES}
%%%%%%%%%%%%%%%%%%%%%%%%%%%%%%%%%%%%%%%%%%%%%%%%%%%%%%%%%%%%%%%%%%%%%%%%%%%%%
% remove this section
% 
Finally check, if your tutor can find
%
\begin{itemize}
  % Generalisierbares Problem
  \item An interesting \emph{problem} (including relevance, context, examples). Possibly, you manage to formulate very specific research questions (RQs).
  \item Your ideas, of how to solve the problem (i.e. you will formulate a \emph{hypothesis}).
  \item If you already found out, possilbe other way to solve the problem (\emph{alternatives} with advantages and disadvantages). 
  \item The steps one needs to solve such a problem and evalutate the solution (i.e. your \emph{method}: for example, will you use questionnaires, perform usability tests, run performance tests, implement a prototype, create a general concept, assemble checklists with best practices, do a literature review, analyse, collect measurements, interpret).
  \item Your \emph{results}, the outcome (i.e. as far as already known, the \emph{main features} of the planned prototype).
  \item Possible (measurable) criteria you might use to evalutate the solution (\emph{metrics}).
  \item Look into the future. Who could use your solution? Possible further research and investigations. 
\end{itemize}
%
}


% Compare: 
% https://sts.univie.ac.at/fileadmin/user_upload/i_sts/Studium/Master_STS/05_Services_for_current_students/Master_Thesis/Guidelines_for_Writing_a_Master_Thesis_Expose.pdf 

% https://tu-dresden.de/ing/informatik/smt/im/studium/theses-and-research-projects/writing-an-expose

% https://www.informatik.hu-berlin.de/de/forschung/gebiete/wbi/teaching/studienDiplomArbeiten/finished/2009/poethig_expose_090402.pdf

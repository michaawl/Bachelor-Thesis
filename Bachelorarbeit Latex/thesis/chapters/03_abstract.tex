%**********************************************************************

%---------------------------------------------------
% NOTE:
% An English version of the abstract is always required 
% (even for German BA/MAs).
%---------------------------------------------------

% right side/flush
\chapterend

\begin{titlepage}

\begin{otherlanguage}{english} 

\begin{abstract} % Abstract
\label{abstract_english}
While gRPC is a well-known and used API approach for service-to-service communication in microservice architecture with high performance requirements, gRPC-Web is still less common on the web. The performance benefits of gRPC, are primarily due to binary Protocol Buffers (Protobuf) serialization format and HTTP/2 multiplexing.
Since most browsers do not support the full set of HTTP/2 features, gRPC-Web cannot fully leverage these advantages. Therefore, REST and GraphQL still remain the dominant web API approaches for web clients.
The goal of the Bachelos thesis is to evaluate how gRPC-Web performs compared to REST and GraphQL with respect to latency, efficiency and resource usage and to identify scenarios in which using gRPC-Web for frontend-backend communication is suitable.
The methodology combines a theoredical analysis with an experimental evaluation using a custom-built prototype. The prototype implements several services that cover typical web payloads (e.g.text or media data) and supports tests both in a browser-based web client and a microservice scenario. The measurement series includes single and parallel requests, cross-browser comparison and a contrast between browser-based frontend-backend communication an service-service communication in the microservice scenario.
Measurements show that, in the browser context, gRPC-Web is not a viable performance-oriented replacement for REST or GraphQL. Especially for large media payloads, REST outperforms both gRPC-Web and GraphQL. Although the theoretical analysis showed that Protocol Buffers more efficient than JSON, these advantages were not measurable. This is due to the limited availability of HTTP/2 features in browsers and the additional proxy/translation layer which is required by gRPC-Web. Another disadvantage compared to REST and GraphQL is, that gRPC-Web is harder to learn and has has less documentation, tools and community support.
Overall gRPC still remains the best option for service-to-service communication in microservice architectures, particularly under high request rates and high performance requirements, the use of gRPC-Web in browsers is just useful, when an existing gRPC backend needs to be exposed to browsers. . Because the gRPC-Web ecosystem is still young, improved browser support for HTTP/2 and HTTP/3 could change this assessment in the future.


\end{abstract}

\end{otherlanguage}


\end{titlepage}


%---------------------------------------------------
% NOTE:
% A German version of the abstract "Zusammenfassung"
% is always required.
%---------------------------------------------------

\begin{titlepage}

\begin{otherlanguage}{german}

\begin{abstract}  % Zusammenfassung
\label{abstract_german}

Während gRPC bereits als etablierter Standard für die Service-zu-Service in Microservice-Architekturen gilt, ist gRPC-Web im Web-Kontext weniger verbreitet. Die Leistungsvorteile von gRPC basieren insbesondere auf der binären Protobuf-Serialisierung und HTTP/2-Multiplexing. Da Browser wichtige HTTP/2-Features nur eingeschränkt unterstützen, kann gRPC-Web diese Vorteile nicht vollständig nutzen. Im Browserumfeld dominieren daher weiterhin REST und GraphQL als etablierte Web-API-Ansätze.
Diese Bachelorarbeit befasst sich mit den Fragen, wie sich gRPC-Web im Vergleich zu den etablierten Web-Technologien REST und GraphQL auf Latenz, Effizienz und Ressourcennutzung in der Frontend-Backend-Kommunikation zwischen einem Web-Client und einem Web-Server auswirkt und unter welchen Bedingungen der Einsatz von gRPC für die Frontend-Backend-Kommunikation sinnvoll ist.
Für die Beantwortung dieser Fragen wird ein kombinierter methodischer Ansatz gewählt: eine theoretischen Analyse und eine experimentelle Evaluation anhand eines eigens entwickelten Prototyps, der die End-zu-End Latenz aus Sicht des Frontend Clients misst. 
Der Prototyp implementiert mehrere Services, die typische Web-Payloads (wie Text- und Mediendaten) abdecken, und Tests sowohl im Browser (Web-Client) als auch in einem Microservice-Szenario (Konsolen-Client). Basierend darauf werden Messreihen mit Einzel- und Parallelenabfragen durchgeführt, verschiedene Browserumbebungen verglichen und die Frontend–Backend-Kommunikation im Browser der Service-zu-Service-Kommunikation im Microservice-Szenario gegenübergestellt.
Die Messungen deuten darauf hin, dass gRPC-Web im Browserumfeld nicht mit den etablierten Web-API-Ansätzen REST und GraphQL konkurrenzfähig ist. Die in der Theorie erwarteten Performanceverbesserungen durch Protobuf konnten durch die Prototypen im Browser nicht bestätigt werden. Grund dafür sind die fehlenden HTTP/2-Features im Browser und der zusätzliche Übersetzungsschritt von gRPC zu gRPC-Web. Zusätzlich ist der Erlernbarkeit von gRPC-Web  im Vergleich zu etablierten Technologien wie REST komplexer und es stehen weniger Dokumentation, Tools und Community-Ressourcen zur Verfügung. Vor allem bei großen Binärdaten erwies sich REST im Browser als robuster.
Insgesamt empfiehlt sich gRPC weiterhin vor allem für Service-zu-Service Szenarien mit hoher Requestfrequenz während eine Implementierung von gRPC-Web nur dann sinnvoll ist, wenn bereits ein gRPC-basiertes Backend besteht. Da gRPC-Web vergleichsweise jung ist , könnte sich dies bei verbesserter Browser-Unterstützung für HTTP/2 oder HTTP/3 künftig ändern. 


\end{abstract}

\end{otherlanguage}

\end{titlepage}

%**********************************************************************

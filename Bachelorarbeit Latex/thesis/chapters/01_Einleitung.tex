%%%%%%%%%%%%%%%%%%%%%%%%%%%%%%%%%%%%%%%%%%%%%%%%%%%%%%%%%%%%%%%%%%%%%%%%%%%%%
\chapter{ Einleitung }
\label{chap:info_REMOVE_ME}
%%%%%%%%%%%%%%%%%%%%%%%%%%%%%%%%%%%%%%%%%%%%%%%%%%%%%%%%%%%%%%%%%%%%%%%%%%%%%
\chapterstart

\section{Motivation}
Bei Echtzeitanwendungen wie Chat-Applikationen oder Streaming Diensten werden
oft große Datenmengen ausgetauscht. Dabei ist eine effiziente Übertragung der Daten
essenziell, damit die Latenz so klein wie möglich bleibt. Eine Methode um dies in
einem Softwareprojekt zu implementieren ist das gRPC Framework, indem die Daten
binär kodiert mittels dem http/2 Protokoll übertragen werden.
Der Hauptanwendungsbereich von gRPC ist die Kommunikation bei sogenannten Microservices
in der Backend Kommunikation, während bei der Kommunikation zwischen
Frontend und Backend meist aufgrund von Kompatibilitäts- und Komplexitätsgründen
REST oder GraphQL Technologien dominieren. Eine gRPC-Übertragung
wird in der Regel nur verwendet, wenn eine effiziente und schnelle Kommunikation
besonders wichtig ist. Ziel der Bachelorarbeit ist es verschieden API - Übertragungstechnologien
vor allem mit dem Fokus auf Latenz, Effizienz und Skalierbarkeit in Bezug
auf Frontend-Backend-Kommunikation zu überprüfen und zu vergleichen.

\section{Forschungsfragen}
Wie wirkt sich die Verwendung von gRPC im Vergleich zu REST
und GraphQL auf die Latenz, Effizienz und Ressourcennutzung in der Frontend-Backend-
Kommunikation aus?

Unter welchen Bedingungen ist der Einsatz von gRPC für die
Frontend-Backend-Kommunikation sinnvoller als REST oder GraphQL?

\section{Hypothese}

Die Nutzung von gRPC reduziert die Latenz und erhöht die Effizienz im
Vergleich zu REST und GraphQL, vor allem bei Echtzeitanwendungen mit hohen Datenmengen.

Die Implementierung von gRPC in React ist komplexer, da dadurch zusätzliche
Tools verwendet werden müssen damit eine Browser Kompatibilität garantiert
werden kann. REST ist geeigneter für einfache Anwendungen mit niedrigeren
Anforderungen an Echtzeitkommunikation.

\section{Methodik}
Theoretische Analyse:
- Untersuchung und Vergleich der Kommunikationsprotokolle gRPC und REST/GraphQL
in Bezug auf Latenz, Effizienz und Skalierbarkeit.
- Analyse von gRPC-Web zur Integration in Frontend-Architekturen (in React).
Erstellen von Prototypen:
Im Zuge der Bachelorarbeit werden Prototypen, mittels den jeweiligen Kommunikationsprotokollen
oder API Gateway erstellt und die Eigenschaften der verschiedenen
Übertragungstechniken verglichen. Dabei werden Performance Tests, Benchmarking,
Skalierbarkeitstests durchgeführt
Verwandte Arbeiten und Stand der Forschung:
Zu den oben genannten Fragestellungen und Technologien sollen verwandte Arbeiten
analysiert und der aktuelle Stand der Forschung erfasst werden.


\chapterend

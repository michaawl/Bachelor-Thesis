%%%%%%%%%%%%%%%%%%%%%%%%%%%%%%%%%%%%%%%%%%%%%%%%%%%%%%%%%%%%%%%%%%%%%%%%%%%%%
\chapter{ Einleitung }
\label{chap:info_REMOVE_ME}
%%%%%%%%%%%%%%%%%%%%%%%%%%%%%%%%%%%%%%%%%%%%%%%%%%%%%%%%%%%%%%%%%%%%%%%%%%%%%
\chapterstart

\section{Motivation}
Bei Echtzeitanwendungen wie Chat-Applikationen oder Streaming Diensten werden
oft große Datenmengen ausgetauscht. Dabei ist eine effiziente Übertragung der Daten
essenziell, damit die Latenz so klein wie möglich bleibt. Eine Methode um dies in 
einem Softwareprojekt zu implementieren ist das gRPC Framework, indem die Daten
binär kodiert mittels dem http/2 Protokoll übertragen werden. Während sich diese Strategie für Datenströme in Backend-Architekturen und insbesondere in Microservice-Umgebungen bereits etabliert haben, ist eine nahtlose Implementierung von gRPC in Frontendanwendungen in Web-Browsern noch nicht möglich. Hauptgrund dafür, sind fehlende HTTP/2-Features in den Browsern (z. B. bidirektionales Streaming) welche eine Voraussetzung für eine Implementierung von gRPC darstellen. 
Eine Abhilfe dafür ist gRPC-Web, eine angepasste Variante von gRPC, die dafür entwickelt wurde, gRPC auch in Webbrowsern nutzbar zu machen.
Während durch die Verwendung von gRPC-Web einige  Features verloren gehen, bleiben auch wichtige Vorteile, wie etwa die Nutzung von Protocol Buffers, erhalten.
In der Praxis wird eine gRPC-Übertragung in der Regel nur verwendet, wenn eine effiziente und schnelle Kommunikation
besonders wichtig ist. Ziel der Bachelorarbeit ist es zu untersuchen, inwiefern sich gRPC beziehungsweise gRPC-Web auch in Frontend-Anwendungen sinnvoll einsetzen lassen. Hierfür werden verschieden API - Übertragungstechnologien, welche sich bereits in der Frontend- zu Backendkommunikation etabliert haben mit gRPC und gRPC-Web
überpüft und verglichen. Der Schwerpunkt liegt auf den Kriterien Latenz, Effizienz und den generellen Vor- und Nachteilen die beim Einsatz der jeweiligen Technologien zum Vorschein kommen. Die Bewertung erfolgt sowohl auf theoretischer Grundlage als auch anhand einer praktischen Untersuchungen.

\section{Forschungsfragen}
Aus der dargestellten Situation und den Einbußen welche mit der Implementierung von gRPC in Webanwendungen einhergehen, ergibt stellt sich die Motivation, die gRPC / gRPC-Web mit den etablierten API-Ansätzen für die Frontend- zu Backendkommunikation zu vergleichen. Um eine diesbezügliche Eignung  von gRPC bzw. gRPC-Web feststellen zu können, werden im Rahmen der Bachelorarbeit folgende Forschungsfragen untersucht:

\begin{itemize}
	\item Wie wirkt sich die Verwendung von gRPC bzw. gRPC-Web im Vergleich zu REST
	und GraphQL auf die Latenz, Effizienz und Ressourcennutzung in der Frontend-Backend-Kommunikation aus?
	
	\item Unter welchen Bedingungen ist der Einsatz von gRPC für die
	Frontend-Backend-Kommunikation sinnvoller als REST oder GraphQL?
	
\end{itemize}

Diese Forschungsfragen bilden die Grundlage für die theoretische Analyse sowie die praktische Untersuchung der genannten API-Technologien. Ziel ist es, anhand dieser Fragen sowohl die Stärken, als auch die Schwächen von gRPC-Web in Webumgebungen zu identifizieren.

\section{Hypothese}
Die formulierten Forschungsfragen geben an, dass vorallem der Vergleich von Latenz und Performance von Interesse ist. Für eine strukturiertere Herangehensweise, werden folgende Hypothese formuliert, die die erwarteten Unterschiede darstellen:

\begin{itemize}
	\item Es wird angenommen, dass die Nutzung von gRPC im Vergleich zu REST und GraphQl die Latenz reduziert und die Effizienz erhöht, insbesondere bei Echtzeitanwendungen mit hohen Datenmengen. Vorallem durch das Verwenden von Protocol Buffers, ein binäres Serialisierungsformat welches auch bei gRPC-Web verwendet wird, ist eine Effizienz im Gegensatz zum klassichen JSON Format zu erwarten, wodurch sich auch beim Datenaustausch bei praktischen Messungen eine geringere Latenz zeigen sollte.
	
	\item Für gRPC-Web wird angenommen, dass sich zwar performancetechnische Vorteile ergeben werden, sich die Implementierung in Projekten der Frontendanwendung jedoch als komplexer gestalten wird. Das Konzept von Protocol Buffers und gRPC ist komplexer als REST/GraphQl und durch den nicht so häufigen Gebrauch in Frontend Anwendungen, gibt es weniger Dokumentation und Diskussionen in Onlineforen. 
	Außerdem wird angenommen, dass eine Vielzahl der Entwickler bereits Vorerfarhungen mit REST bzw. Kenntnisse zu http Methoden  gesammelt haben.

	
\end{itemize}

\section{Methodik}
Theoretische Analyse:
- Untersuchung und Vergleich der Kommunikationsprotokolle gRPC und REST/GraphQL
in Bezug auf Latenz, Effizienz und Skalierbarkeit.
- Analyse von gRPC-Web zur Integration in Frontend-Architekturen (in React).
Erstellen von Prototypen:
Im Zuge der Bachelorarbeit werden Prototypen, mittels den jeweiligen Kommunikationsprotokollen
oder API Gateway erstellt und die Eigenschaften der verschiedenen
Übertragungstechniken verglichen. Dabei werden Performance Tests, Benchmarking,
Skalierbarkeitstests durchgeführt
Verwandte Arbeiten und Stand der Forschung:
Zu den oben genannten Fragestellungen und Technologien sollen verwandte Arbeiten
analysiert und der aktuelle Stand der Forschung erfasst werden.


\chapterend

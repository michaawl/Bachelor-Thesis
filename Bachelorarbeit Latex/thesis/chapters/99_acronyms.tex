
%%%%%%%%%%%%%%%%%%%%%%%%%%%%%%%%%%%%%%%%%%%%%%%%%%%%%%%%%%%%%%%%%%%%%%%%%%%%%
% Note 1: the * with \chapter*, which hides it from TOC. 
% Note 2: \thispagestyle{empty} suppresses page number on the first page
%         i.e. to be consistent with the other (numbered) chapters.
\chapter*{Akronyme\thispagestyle{empty}} 
\label{chap:acronyms}
%%%%%%%%%%%%%%%%%%%%%%%%%%%%%%%%%%%%%%%%%%%%%%%%%%%%%%%%%%%%%%%%%%%%%%%%%%%%%


% In your text use macro \ac all the time.
%   E.g.  \ac{MITM}
% Note for pretty printing the list of acronyms:
%   First, find out which one will be the longest (here e.g. KISS or MITM).
%   Then, specify as many chars (e.g. 4 Ms) such as \begin{acronym}[MMMM].
\footnotesize
\begin{acronym}[MMMM]

  % MUST be sorted manually:
  \acro{Protobuf}{Protocol Buffers}
  \acro{JSON}{JavaScript Object Notation}
  \acro{REST}{Representational State Transfer}
  \acro{GraphQL}{Graph Query Language}
  \acro{gRPC}{gRPC Remote Procedure Call}
  \acro{HTTP}{Hypertext Transfer Protocol}
  \acro{API}{Application Programming Interface}
  \acro{Blob}{Binary Large Object}
  \acro{HTTPS}{Hypertext Transfer Protocol Secure}
  \acro{RPC}{Remote Procedure Call}

  

  %
  % You get warnings for unused acronyms, so better disable them
  %
  %\acro{ACL} {Access Control List}
  %\acro{GUI} {Graphical User Interface}
  %\acro{KISS}{Keep It Small and Simple}
  %\acro{OS}  {Operating System}
  %\acro{UART}{Universal Asynchronous Receiver/Transmitter}
  %\acro{UID} {Unique Identifier}

\end{acronym}
\normalsize
